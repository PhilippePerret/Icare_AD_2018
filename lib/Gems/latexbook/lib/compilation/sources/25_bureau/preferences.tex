% <!--
%   CE FICHIER CONTIENT LES DÉFINITIONS GÉNÉRALES DES LIENS
% 
%   DÉFINITION :
% 
%   [un texte identifiant]:  une/path/to/cible  "Le titre optionnel"
% 
%   UTILISATION
% 
%   avec le texte identique :
% 
% un texte identifiant][
% 
%   avec un autre texte :
% 
% autre texte pour le lien][un texte identifiant
% 
%   -->

\chapter{Préférences}\hypertarget{preferences-auteur}{}\label{preferences-auteur}

Cette section décrit toutes vos préférences pour le programme \unan{} et comment les régler.

\section{Réglage de vos préférences}\hypertarget{reglage-preferences}{}\label{reglage-preferences}

\subsection{Régler le rythme du programme}\hypertarget{regler-rythme}{}\label{regler-rythme}

\begin{itemize}
\item \hyperlink{rejoindre-site}{Rejoindre le site},
\item \hyperlink{vous-identifier}{Vous identifier},
\item \hyperlink{rejoindre-bureau}{Rejoindre votre bureau},
\item \activerPanneauBureau{Préférences},
\item dans la partie \enquote{Programme}, réglez la vitesse du menu \enquote{Rythme} (\emph{avec le rythme \enquote{moyen}, un \hyperlink{explicationjourprogrammejourreel}{jour-programme} correspond à un jour réel}),
\item cliquer le bouton \enquote{Enregistrer} pour enregistrer les modifications.
\end{itemize}

Votre nouveau rythme de travail prend effet dès à présent.

\subsection{Rejoindre son bureau après connexion}\hypertarget{rejoindre-bureau-apres-connexion}{}\label{rejoindre-bureau-apres-connexion}

Vous pouvez régler vos préférences pour rejoindre votre bureau de travail, le centre névralgique du développement de votre histoire, tout de suite après vous être identifié\fem{e}. Pour ce faire, procéder comme suit~{}:

\begin{itemize}
\item \hyperlink{rejoindre-site}{Rejoindre le site},
\item \hyperlink{vous-identifier}{Vous identifier},
\item \hyperlink{rejoindre-bureau}{Rejoindre votre bureau},
\item \activerPanneauBureau{Préférences},
\item dans la partie \enquote{Navigation}, cochez —~{}ou décochez~{}— la case \enquote{Rejoindre le bureau après l'identification},
\item cliquer le bouton \enquote{Enregistrer} pour prendre en compte les modifications.
\end{itemize}

La prochaine fois que vous vous connecterez, vous serez directement redirigé\fem{e} vers votre bureau du programme \unan{}.

\subsection{Régler l'affichage de la pastille de tâches}\hypertarget{regler-pastille-taches}{}\label{regler-pastille-taches}

La pastille de tâches, en haut à droite de votre fenêtre, vous permet de connaitre rapidement votre situation au niveau des travaux en dépassement, nouveaux, à démarrer, etc. Vous pouvez, grâce aux préférences, afficher ou masquer cette pastille.

\begin{itemize}
\item \hyperlink{rejoindre-site}{Rejoindre le site},
\item \hyperlink{vous-identifier}{Vous identifier},
\item \hyperlink{rejoindre-bureau}{Rejoindre votre bureau},
\item \activerPanneauBureau{Préférences},
\item dans la partie \enquote{Notifications}, cocher la case \enquote{Pastille taches dans l'entête de page} pour afficher la pastille ou la décocher pour la masquer,
\item cliquer le bouton \enquote{Enregistrer} pour enregistrer les modifications.
\end{itemize}

\subsection{Régler l'envoi du mail quotidien}\hypertarget{regler-mail-quotidien}{}\label{regler-mail-quotidien}

Deux préférences vous permettent de décider de la fréquence de réception du mail rapport et de l'heure de réception.

\begin{itemize}
\item \hyperlink{rejoindre-site}{Rejoindre le site},
\item \hyperlink{vous-identifier}{Vous identifier},
\item \hyperlink{rejoindre-bureau}{Rejoindre votre bureau},
\item \activerPanneauBureau{Préférences},
\item dans la partie \enquote{Notifications}, cocher la case \enquote{Récapitulatif journalier} pour recevoir quotidiennement votre état concernant le programme, même lorsqu'il n'y a pas de nouveaux travaux. Décocher au contraire cette case si l'on ne veut recevoir les travaux que lorsqu'il y a de nouveaux travaux ou que certaines tâches sont en dépassement,
\item dans la même partie, cocher la case \enquote{Envoyer le mail quotidien à} et régler l'heure voulue pour recevoir le mail à une heure définie,
\item cliquer le bouton \enquote{Enregistrer} pour enregistrer les modifications.
\end{itemize}

\subsection{Régler le partage du projet}\hypertarget{regler-partage-projet}{}\label{regler-partage-projet}

Grâce aux préférences, vous pouvez définir les personnages qui peuvent suivre —~{}ou pas~{}— votre projet.

\begin{itemize}
\item \hyperlink{rejoindre-site}{Rejoindre le site},
\item \hyperlink{vous-identifier}{Vous identifier},
\item \hyperlink{rejoindre-bureau}{Rejoindre votre bureau},
\item \activerPanneauBureau{Préférences},
\item dans la partie \enquote{Partage}, réglez le menu pour choisir qui peut suivre votre projet,
\item cliquer le bouton \enquote{Enregistrer} pour enregistrer les modifications.
\end{itemize}
