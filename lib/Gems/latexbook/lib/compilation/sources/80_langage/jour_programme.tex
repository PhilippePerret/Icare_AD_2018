% <!--
%   CE FICHIER CONTIENT LES DÉFINITIONS GÉNÉRALES DES LIENS
% 
%   DÉFINITION :
% 
%   [un texte identifiant]:  une/path/to/cible  "Le titre optionnel"
% 
%   UTILISATION
% 
%   avec le texte identique :
% 
% un texte identifiant][
% 
%   avec un autre texte :
% 
% autre texte pour le lien][un texte identifiant
% 
%   -->

\section{Jour-programme}\hypertarget{explicationjourprogrammejourreel}{}\label{explicationjourprogrammejourreel}

Quand on suit le programme \unan{}, il faut comprendre qu'il y a deux sortes de jours~{}: les ``jours-programme'' et les ``jours réels''. Lorsque l'on suit le programme en rythme \emph{normal}, c'est-à-dire en rythme \emph{moyen}, un jour-programme et un jour réel sont équivalents ou, pour le dire autrement, un jour-programme dure la durée d'un jour réel, c'est-à-dire \emph{24 heures}.

Le programme \unan{} étant pensé pour se réaliser en une année, il est décomposé en \textbf{366 jours-programme} qui tous ensemble définissent le programme complet.

Lorsque l'on ralentit son rythme de travail, les jours-programme allongent leur durée, excédant celle d'un jour réel —~{}on peut alors faire le programme \unan{} en un an et demi par exemple. Mas le \emph{nombre} de jours-programme, lui, ne varie jamais, il y en aura toujours 366 dans le programme, même si ce programme est réalisé en plus d'un an.

Si, par exemple, vous suivez le programme \unan{} avec un rythme deux fois plus lent que le rythme moyen, vous mettrez deux fois plus de temps pour le réaliser, donc deux années (cela peut vous sembler long, en réalité, il faut bien plus de temps que ça à un auteur pour apprendre la dramaturgie et réussir son premier scénario professionnel~{}! La première qualité de l'auteur doit être la \emph{patience}). Deux ans pour suivre le programme prévu sur une année, cela signifie concrètement qu'un \emph{jour-programme} durera alors \emph{2 jours réels}.

