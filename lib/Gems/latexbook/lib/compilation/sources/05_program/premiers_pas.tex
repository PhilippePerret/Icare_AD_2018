% <!--
%   CE FICHIER CONTIENT LES DÉFINITIONS GÉNÉRALES DES LIENS
% 
%   DÉFINITION :
% 
%   [un texte identifiant]:  une/path/to/cible  "Le titre optionnel"
% 
%   UTILISATION
% 
%   avec le texte identique :
% 
% un texte identifiant][
% 
%   avec un autre texte :
% 
% autre texte pour le lien][un texte identifiant
% 
%   -->

\section{Premiers pas dans le programme}\hypertarget{premiers-pas}{}\label{premiers-pas}

\subsection{Inscription au programme}\hypertarget{inscription-programme}{}\label{inscription-programme}

La toute première chose à faire pour débuter avec le programme \unan{} est de s'inscrire à ce programme sur le site de \boa{}. Si vous êtes déjà inscrit\fem{e}, vous pouvez passer cette partie pour \hyperlink{premiers-pas-apres-inscription}{poursuivre cette présentation des premiers pas}.

Deux procédures d'inscription au programme sont possibles suivant que~{}:

\begin{itemize}
\item vous êtes déjà inscrit\fem{e} sur le site ou même abonné\fem{e}~{}: suivez la \hyperlink{procedure-inscription-rapide}{procédure d'inscription rapide au programme}~{};
\item vous n'êtes ni inscrit\fem{e} ni abonné\fem{e} à \boa{}~{}: suivez la \hyperlink{procedure-inscription-complete}{procédure d'inscription complète au programme}.
\end{itemize}

\subsection{Procédure d'inscription rapide au programme}\hypertarget{procedure-inscription-rapide}{}\label{procedure-inscription-rapide}

\begin{itemize}
\item \hyperlink{rejoindre-site}{rejoindre le site},
\item \hyperlink{vous-identifier}{vous identifier},
\item cliquer sur le lien ``1 un 1 script'' dans la marge gauche,
\item cliquer sur le lien ``s'inscrire au programme'' qui s'affiche dans la page,
\item remplir et soumettre le formulaire d'inscription qui s'affiche.
\end{itemize}

Sitôt l'inscription et le paiement accepté, vous êtes inscrit\fem{e} au programme et vous pouvez d'ores et déjà le commencer~{}!

\subsection{Procédure d'inscription complète au programme}\hypertarget{procedure-inscription-complete}{}\label{procedure-inscription-complete}

\begin{itemize}
\item \hyperlink{rejoindre-site}{rejoindre le site},
\item cliquer sur le bouton ``s'inscrire'' dans la marge gauche,
\item sur le formulaire qui s'affiche, \textbf{cocher la case ``Inscription au programme Un an Un script''},
\item finir de remplir le formulaire et le soumettre.
\end{itemize}

Sitôt l'inscription et le paiement accepté, vous êtes inscrit\fem{e} au programme et vous pouvez d'ores et déjà le commencer.

\subsection{Après l'inscription}\hypertarget{premiers-pas-apres-inscription}{}\label{premiers-pas-apres-inscription}

Après l'inscription, le premier travail que vous ayez à faire est très simple, il consiste à fournir quelques renseignements sur le projet que vous allez développer.

Rassurez-vous, ces renseignements sont personnels et vous pourrez les modifier dès que vous le souhaitez, sans aucune limitation.

Pour ce faire, rejoignez votre bureau —~{}simplement en cliquant sur ``1 an 1 script'' dans votre marge gauche si vous êtes encore connecté\fem{e} ou en suivant la \hyperlink{rejoindre-bureau-travail}{procédure de connexion à son bureau} si ça n'est plus le cas. Ensuite~{}:

\begin{itemize}
\item \activerPanneauBureau{Projet},
\item donner \textbf{un titre de travail} à votre projet,
\item donner \textbf{un résumé} si vous l'avez déjà ou indiquer simplement la mention ``résumé à venir'',
\item indiquer \textbf{le type de support} (film, roman, etc.),
\item indiquer \textbf{le niveau de partage} de ce projet. Ce niveau de partage déterminera qui pourra suivre votre travail dans le programme (``suivre'', ici, signifie que ces personnes ne pourront voir que ce que vous déciderez de leur montrer).
\item cliquer enfin sur le bouton ``Enregistrer'' pour prendre en compte ces informations.
\end{itemize}

Bravo~{}! Vous venez d'accomplir le premier travail de votre tout premier \hyperlink{explicationjourprogrammejourreel}{jour-programme}.

\subsection{Poursuivre le premier jour}\hypertarget{poursuivre-le-premier-jour}{}\label{poursuivre-le-premier-jour}

Pour poursuivre ce premier jour, en attendant le travail qui vous attendra demain, vous pouvez accomplir plusieurs petites choses.

La première peut consister à consulter ce document pour en apprendre plus sur le programme.

Vous pouvez, en particulier, vous informer sur une notion importante~{}: le rythme du programme.

