% <!--
%   CE FICHIER CONTIENT LES DÉFINITIONS GÉNÉRALES DES LIENS
% 
%   DÉFINITION :
% 
%   [un texte identifiant]:  une/path/to/cible  "Le titre optionnel"
% 
%   UTILISATION
% 
%   avec le texte identique :
% 
% un texte identifiant][
% 
%   avec un autre texte :
% 
% autre texte pour le lien][un texte identifiant
% 
%   -->

\section{Foire aux questions}\hypertarget{faq}{}\label{faq}

\begin{description}
\item[Quelle est la différence entre un module à l'\href{http://www.atelier-icare.net}{Atelier Icare} et ce programme~{}?] \hfill \\
 Hormis le fait que la pédagogie employée est la même, les deux choses sont foncièrement différentes. À l'atelier Icare, on est accompagné de près, on est suivi le long du développement de son projet. On est en rapport direct, en discussion, avec Philippe Perret, le pédagogue. À l'opposé, le programme UN AN UN SCRIPT est pensé pour offrir une autonomie totale à l'auteur. Il reste seul, mais il n'est pas livré à lui-même, il suit une sorte de route éclairée par un guide pédagogue.



Bien entendu, la différence se fait également au niveau du tarif de l'un et de l'autre, qui sont en rapport de leur différence. Le programme UN AN UN SCRIPT a vraiment été pensé pour être accessible à toutes les bourses alors que l'\href{http://www.atelier-icare.net}{atelier Icare} est un véritable investissement financier.



\item[Est-ce que l'inscription au site vaut inscription au programme~{}?] \hfill \\
 Non, l'inscription au site \boa{} et l'inscription au programme \unan{} sont deux choses bien distinctes.



De la même manière, l'abonnement au site permet d'utiliser l'intégralité des outils mais ne permet pas de suivre le programme.



En d'autres termes, le programme \unan{} est un outil particulier et tout à fait à part sur le site.



\item[Puis-je mettre mon programme en pause~{}?] \hfill \\
 Pour le moment, la procédure n'est pas possible, mais si vous insistez auprès de l'administration, un arrangement devrait toujours être possible ;-).



\item[Puis-je soumettre mon travail à quelqu'un~{}?] \hfill \\
 Oui, tout à fait, et c'est même conseillé, en utilisant \leForum{} de l'atelier.



\item[Peut-on contacter d'autres auteurs du programme~{}?] \hfill \\
 Tout à fait~{}! À partir du moment où ils ont réglé leurs \hyperlink{preferences-auteur}{préférences} pour permettre d'être contacté par d'autres auteurs du programme.



\item[Où peut-on joindre Philippe Perret~{}?] \hfill \\
 Vous pouvez le joindre en utilisant son mail~{}: \mailphil{phil@laboiteaoutilsdelauteur.fr} ou en passant par l'\mailadmin{administration}.



\item[Dans le message, il est stipulé : Aucun travail en retard. Donc rien de précis à faire pour le moment ?] \hfill \\
 Chaque travail à faire possède une échéance précise (1 jour, 10 jours, 3 mois…). Un travail est en retard lorsque cette échéance est dépassée. Donc, l'indication “aucun travail en retard” précise seulement qu'il n'y a aucun travail en dépassement d'échéance, mais elle indique aussi forcément, par induction, qu'il y a des travaux en cours ;-).
\end{description}

Bien entendu, si vous avez d'autres questions, n'hésitez surtout pas à \href{http://www.laboiteaoutilsdelauteur.fr/site/contact}{prendre contact avec Phil pour lui poser}~{}!

