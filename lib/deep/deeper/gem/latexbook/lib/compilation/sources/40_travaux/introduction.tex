% <!--
%   CE FICHIER CONTIENT LES DÉFINITIONS GÉNÉRALES DES LIENS
% 
%   DÉFINITION :
% 
%   [un texte identifiant]:  une/path/to/cible  "Le titre optionnel"
% 
%   UTILISATION
% 
%   avec le texte identique :
% 
% un texte identifiant][
% 
%   avec un autre texte :
% 
% autre texte pour le lien][un texte identifiant
% 
%   -->

\chapter{Les Travaux}\hypertarget{les-travaux}{}\label{les-travaux}

\section{Introduction aux travaux}\hypertarget{introduction-aux-travaux}{}\label{introduction-aux-travaux}

Les \enquote{travaux} sont toutes les actions que vous avez à accomplir au cours du programme “UN AN UN SCRIPT”. Parmi ces travaux, on trouve pêle-mêle~{}:

\begin{itemize}
\item lire une page de cours~{};
\item rédiger des documents de travail concernant votre projet~{};
\item répondre à un questionnaire de renseignement~{};
\item faire un QCM de validation des acquis~{};
\item proposer un document en lecture sur le forum~{};
\item répondre à des messages sur le forum~{};
\item analyser un de vos travaux grâce aux outils proposés~{};
\item etc.
\end{itemize}

Cette liste n'est pas exhaustive et de nombreux autres types de travaux peuvent être accomplis pour atteindre le but du programme unan\{\}~{}: la rédaction de la deuxième version d'un scénario ou d'un manuscrit.

Voyons plus précisément quels sont les types de tâches que vous pouvez trouver et ce qu'elles recouvrent comme travail.

