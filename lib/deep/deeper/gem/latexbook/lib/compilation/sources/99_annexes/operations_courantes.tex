% <!--
%   CE FICHIER CONTIENT LES DÉFINITIONS GÉNÉRALES DES LIENS
% 
%   DÉFINITION :
% 
%   [un texte identifiant]:  une/path/to/cible  "Le titre optionnel"
% 
%   UTILISATION
% 
%   avec le texte identique :
% 
% un texte identifiant][
% 
%   avec un autre texte :
% 
% autre texte pour le lien][un texte identifiant
% 
%   -->

\chapter{Opérations courantes}\hypertarget{operations-courantes}{}\label{operations-courantes}

Cette section présente les opérations de base du programme \unan{} qu'il est bon de connaitre, mais cette section est là aussi pour vous permettre de les retrouver facilement.

\section{Rejoindre le site}\hypertarget{rejoindre-le-site}{}\label{rejoindre-le-site}

Pour rejoindre le site, tapez l'adresse~{}: \url{\textbackslash{}urlsite\{}\}\} dans la barre de navigation de votre navigateur favori.

\section{S'identifier au site}\hypertarget{vous-identifier}{}\label{vous-identifier}

Si vous êtes inscrit\fem{e} ou abonné\fem{e} au site, pour vous identifier~{}:

\begin{itemize}
\item \hyperlink{rejoindre-le-site}{rejoindre le site},
\item cliquer sur le menu ``S'identifier'' dans la marge gauche,
\item remplir le formulaire qui s'affiche en indiquant 1/ votre mail (celui avec lequel vous vous êtes inscrit\fem{e}) et votre mot de passe (idem),
\item cliquer enfin sur le bouton ``OK''.
\end{itemize}

Si vous n'avez pas commis d'erreur, vous devez être redirigé\fem{e} vers votre bureau de travail ou votre profil suivant vos \hyperlink{preferences-auteur}{préférences}.

\section{Rejoindre son bureau de travail}\hypertarget{rejoindre-bureau-travail}{}\label{rejoindre-bureau-travail}

Pour rejoindre votre bureau de travail sur le site de \boa{}, il suffit de~{}:

\begin{itemize}
\item \hyperlink{rejoindre-le-site}{rejoindre le site},
\item \hyperlink{vous-identifier}{vous identifier},
\item Si vous avez choisi l'option ``rejoindre votre bureau après identification'' dans vos \hyperlink{preferences-auteur}{préférences}, alors vous rejoignez aussitôt votre bureau de travail,
\item Sinon, cliquer sur le lien ``1 an 1 script'' qui se trouve dans la marge gauche du site.
\end{itemize}
