% <!--
%   CE FICHIER CONTIENT LES DÉFINITIONS GÉNÉRALES DES LIENS
% 
%   DÉFINITION :
% 
%   [un texte identifiant]:  une/path/to/cible  "Le titre optionnel"
% 
%   UTILISATION
% 
%   avec le texte identique :
% 
% un texte identifiant][
% 
%   avec un autre texte :
% 
% autre texte pour le lien][un texte identifiant
% 
%   -->

\section{Aperçu rapide du programme}\hypertarget{overview}{}\label{overview}

Une fois inscrit\fem{e} au programme \unan{}, vous entrez dans un processus de développement de projet et d'apprentissage de la narration. Cet apprentissage vous conduit, en une année-programme, à parachever la deuxième version d'un scénario ou d'un manuscrit et terminer un cycle complet d'enseignement théorique et pratique de la dramaturgie, art de transformer une \emph{histoire} en \emph{récit}.

Dès l'inscription un \bureau{} vous est attribué sur le site \boa{}, qui est votre centre de travail principal. Vous y trouvez tout ce qu'il faut pour suivre votre programme.

Quotidiennement —~{}ou suivant vos \hyperlink{preferences-auteur}{préférences}~{}—, vous recevez un mail vous informant très précisément du travail que vous devez accomplir dans la journée, dans la semaine ou dans les semaines à venir.

Des exemples vous permettent de savoir et comprendre très exactement ce que vous devez accomplir, vous ne restez jamais dans le flou.

D'étape en étape, vous avancez dans la réalisation de votre projet et vous apprenez en parallèle les secrets de l'écriture et de la conception d'un récit. Vous vous familiarisez également avec une méthode de conception souple que vous pourrez adapter aisément à vos préférences.

En parallèle, vous pouvez participer au \forum{} et poser toutes vos questions. Des auteurs professionnels, à commencer par le créateur de ce programme, Philippe Perret, sont là pour vous répondre du mieux qu'ils le peuvent.

