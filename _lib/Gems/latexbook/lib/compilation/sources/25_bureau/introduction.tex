% <!--
%   CE FICHIER CONTIENT LES DÉFINITIONS GÉNÉRALES DES LIENS
% 
%   DÉFINITION :
% 
%   [un texte identifiant]:  une/path/to/cible  "Le titre optionnel"
% 
%   UTILISATION
% 
%   avec le texte identique :
% 
% un texte identifiant][
% 
%   avec un autre texte :
% 
% autre texte pour le lien][un texte identifiant
% 
%   -->

\chapter{Le bureau de travail}\hypertarget{bureautravail}{}\label{bureautravail}

\section{Introduction}\hypertarget{introduction-bureau}{}\label{introduction-bureau}

Le \textbf{bureau de travail} est le centre névralgique des opérations d'un\fem{e} auteur\fem{e} inscrit\fem{e} au programme \unan{} sur le site \boa{}.

\imagecenter[scale=0.5]{ui/onglets-bureau.png}

Ce bureau vous permet de suivre très précisément votre progression tout au long du programme, de connaitre les tâches à exécuter, de répondre aux questionnaires de validation des acquis, de répondre aux messages du forum également.

Ce bureau est composé de différents onglets que nous allons passer en revue rapidement.

\begin{description}
\item[Onglet ``État''] \hfill \\
 Il donne un aperçu général de votre programme à commencer par votre \hyperlink{explicationjourprogrammejourreel}{jour-programme} courant et votre nombre de points.



\item[Onglet ``Projet''] \hfill \\
 Il vous permet de définir les données générales de votre projet, à commencer par son titre et son résumé.



\item[Onglet ``Tâches''] \hfill \\
 Cet onglet définit la liste des tâches que vous avez à accomplir ou que vous venez d'accomplir.



\item[Onglet ``Cours''] \hfill \\
 Onglet qui liste les pages de cours que vous devez lire ou celles que vous venez tout juste de lire.



\item[Onglet ``Forum''] \hfill \\
 Définit les choses à faire par rapport aux messages forum, aux invitations ou aux éventuelles questions relatives à votre projet.



\item[Onglet ``Quiz''] \hfill \\
 Contient les questionnaires que vous devez remplir ou ceux que vous venez de remplir.



\item[Onglet ``Préférences''] \hfill \\
 Vous permet de régler vos préférences pour la conduite du programme.



\item[Onglet ``Aide''] \hfill \\
 Pour obtenir de l'aide, à commencer par ce fichier PDF qui vous permet de consulter tranquillement le manuel d'utilisation de votre programme.
\end{description}

