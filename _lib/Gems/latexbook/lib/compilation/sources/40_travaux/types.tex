% <!--
%   CE FICHIER CONTIENT LES DÉFINITIONS GÉNÉRALES DES LIENS
% 
%   DÉFINITION :
% 
%   [un texte identifiant]:  une/path/to/cible  "Le titre optionnel"
% 
%   UTILISATION
% 
%   avec le texte identique :
% 
% un texte identifiant][
% 
%   avec un autre texte :
% 
% autre texte pour le lien][un texte identifiant
% 
%   -->

\section{Les types de travaux}\hypertarget{type-travaux}{}\label{type-travaux}

Les “travaux” du programme \unan{} sont classés en trois grands types~{}:

\begin{description}
\item[Les lectures de pages de cours] \hfill \\
 Parmi ces pages de cours à lire ou à relire, il existe deux types de pages~{}:  celles qui appartiennent à la collection Narration, le cours dramaturgique associé au programme et celles qui sont propres au programme et concernent le plus souvent des éléments très concrets du travail.



On pourrait ajouter un troisième type de page qui est une extension des pages de la collection Narration ou les exemples qui sont donnés pour les travaux du programme.



\item[Les questionnaires et autres quiz] \hfill \\
 Ils permettent notamment de valider vos acquis au fil du programme, ou de procéder à des checks réguliers de votre projet. Ce sont des QCM qui permettent d'obtenir de bons points en fonction des résultats.



Le plus souvent, ces questionnaires sont destinés à valider les lectures que vous avez faites des pages de cours ou à tester la pertinence des tâches que vous avez accomplies.



\item[Les actions forum] \hfill \\
 Elles concernent le forum du site, forum qui doit apporter une aide plus personnelle au suivi du programme. Ces actions peuvent consister à contacter un participant pour obtenir une lecture critique d'un document, à répondre à une critique formulée ou même à proposer un nouveau sujet par rapport à votre propre expérience au sein du programme.



\item[Les tâches] \hfill \\
 Ce sont tous les travaux qui concernent en général très directement le projet que vous développez au sein du programme. La tâche type est la rédaction d'un document de travail, synopsis, résumé, traitement et, au bout du compte, le scénario ou le manuscrit. Mais ces tâches peuvent être beaucoup plus variées, depuis la documentation jusqu'à l'analyse d'un de vos propres textes ou d'un film que vous appréciez.



Ces tâches représentent l'essence même du programme.
\end{description}

\subsection{Onglets de travaux}\hypertarget{onglets-de-travaux}{}\label{onglets-de-travaux}

Sur votre \hyperlink{bureau}{bureau de travail}, les différents types de travaux sont répartis en plusieurs onglets distincts~{}: l'onglet ``Tâches'', l'onglet ``Cours'', l'onglet ``Forum'' et l'onglet ``Quiz''.

Dès qu'un type contient des travaux, le nombre de ces travaux est indiqué dans l'onglet lui-même, après le nom. Ce nombre est en rouge si un travail n'a pas été démarré.

\subsection{Contenu des cadres de travaux}\hypertarget{contenu-des-cadres-de-travaux}{}\label{contenu-des-cadres-de-travaux}

Comme vous l'avez peut-être déjà vu, chaque travail est représenté dans un cadre qui permet d'en prendre connaissance et de connaitre son état.

Pour une page de cours, il s'agit d'un simple résumé de la page. Un bouton permet de l'afficher pour pouvoir la lire.

\imagecenter[scale=0.5]{travaux/cadre-page-cours.png}

la date d'échéance de lecture est indiquée le plus souvent, un bouton permet de préciser au programme que la page a été lue et un autre permet de l'insérer dans sa table des matières personnelle, afin de la retrouver plus facilement.

Pour un questionnaire ou un quiz, après l'avoir démarré, le cadre de travail affiche tout simplement le questionnaire lui-même.

Les cadres de travaux les plus complexes concernent les \emph{tâches}. Voyons ces cadres en détail.

\imagecenter[scale=0.5]{travaux/cadre-travail.png}

Ces cadres contiennent~{}:

\begin{itemize}
\item le \textbf{titre} du travail, le résumant,
\item en regard de ce titre, on trouve le nombre de \textbf{points} qui seront acquis à la fin de ce travail, c'est la partie ludique du programme,
\item une \textbf{description} précise de ce travail, qui doit contenir l'intitulé de ce qu'il convient de produire,
\item un \textbf{cadre d'information temporelle} qui indique quand le travail a débuté, quand il doit être achevé, la date d'échéance exacte et le nombre de jours restants. En cas de dépassement, ce cadre est rouge et indique le nombre de jours de dépassement,
\item une \textbf{bande de boutons} permettant notamment de démarrer le travail et d'indiquer quand il est terminé,
\item une bande de renseignements généraux, juste pour information,
\end{itemize}

Sous cette bande de renseignements généraux on peut trouver, à titre optionnel~{}:

\begin{itemize}
\item l'indication de pages de cours à lire,
\item des boutons permettant d'afficher les exemples qui aideront à la rédaction de certains documents,
\item exceptionnellement, des tutoriels-vidéos pour les tâches les plus compliquées à réaliser, concernant le plus souvent des manipulations sur l'ordinateur ou le site.
\end{itemize}
