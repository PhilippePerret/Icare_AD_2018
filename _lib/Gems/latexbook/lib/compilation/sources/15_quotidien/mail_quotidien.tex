% <!--
%   CE FICHIER CONTIENT LES DÉFINITIONS GÉNÉRALES DES LIENS
% 
%   DÉFINITION :
% 
%   [un texte identifiant]:  une/path/to/cible  "Le titre optionnel"
% 
%   UTILISATION
% 
%   avec le texte identique :
% 
% un texte identifiant][
% 
%   avec un autre texte :
% 
% autre texte pour le lien][un texte identifiant
% 
%   -->

\section{Mail quotidien}\hypertarget{mail-quotidien}{}\label{mail-quotidien}

… ``ou pas [quotidien]'' devrions-nous ajouter au titre, car ce mail dont nous allons parler n'est quotidien que pour deux raisons~{}:

\begin{itemize}
\item votre \emph{rythme de travail}] (\ref{rythme-travail}) est \emph{moyen} (\emph{normal}) et vous changez donc de \emph{jour-programme} (\ref{explicationjourprogrammejourreel}) tous les jours~{};
\item votre rythme est plus lent que le rythme \emph{normal} mais vous avez réglé vos \preferences{} de telle sorte que vous recevez ce message quotidiennement.
\end{itemize}

Dans tous les autres cas, vous ne recevez ce message qu'au moment de votre passage d'un jour-programme à un autre lorsqu'il y a des dépassements ou de nouveau travaux.

\subsection{État des lieux de votre programme}\hypertarget{tat-des-lieux-de-votre-programme}{}\label{tat-des-lieux-de-votre-programme}

Ce mail quotidien —~{}nous l'appellerons ainsi même s'il n'est pas forcément journalier~{}— vous présente un \textbf{état des lieux de votre programme}, vous indiquant vos nouveaux travaux, vos travaux en retard, vos travaux à démarrer ainsi que l'échéance des travaux qui se poursuivent et toute information qui peut vous être utile en temps donné.

Il peut ressembler à ça~{}:

\begin{center}
\image[scale=0.6]{mail/mail-quotidien-1.png}
\image[scale=0.6]{mail/mail-quotidien-2.png}
\end{center}

\subsection{Contenu du mail}\hypertarget{contenu-du-mail}{}\label{contenu-du-mail}

Dans le détail et dans son intégralité, ce mail quotidien contient ces différentes sections~{}:

\begin{description}
\item[Un titre] \hfill \\
 Ce titre vous précise la date du rapport qui vous est fait.



\item[Une invite générale] \hfill \\
 Le mail commence par un message de salutation générale, donnant en quelque sorte la \emph{température} du message. Cette invite, par exemple, peut vous mettre en garde contre d'éventuels retards à répétition.



\item[Un résumé] \hfill \\
 Ensuite, vous trouvez un résumé général de votre état, contenant notamment votre numéro de jour-programme, votre nombre de points ainsi qu'une note générale concernant votre jour précédant.



Cette note dépend en fait de vos dépassements d'échéance ou de vos oublis de démarrage de travaux. Hé oui, le programme n'est pas méchant, mais il tente de vous maintenir du mieux qu'il peut dans le droit chemin ;-).



\item[Travaux en dépassement d'échéance] \hfill \\
 En dessous du résumé, et seulement si vous avez des échéances dépassées, vous trouvez la liste des travaux qui auraient dû être achevés, avec l'indication du nombre de jours de retard.



Si ce nombre de dépassement augmente trop, pensez à \textbf{adapter votre rythme à vos possibilités}. En réduisant le rythme, vous aurez d'autant plus le temps pour réaliser vos tâches.



\item[Travaux à démarrer] \hfill \\
 Vient un cadre contenant les travaux à démarrer, si vous en avez oubliés la veille.



Peut-être en aurez-vous souvent au début, mais vous prendrez vite l'habitude de rejoindre votre bureau chaque jour-programme pour démarrer toutes vos nouvelles tâches, et donc ne pas vous faire réprimander par le programme~{}!



\item[Nouveaux travaux] \hfill \\
 Vous devriez ensuite trouver un cadre contenant les nouveaux travaux à accomplir. Au début du programme, vous trouverez chaque jour de nouveaux travaux, mais plus vous avancerez et moins il y en aura. Cela tient simplement au fait que plus le développement avance et plus long sont les documents à produire, donc plus longue est leur échéance.



Rappelez-vous, comme indiqué ci-dessus~{}: vous devez toujours informer le programme que vous avez pris en compte ces nouveaux travaux.



\item[Travaux poursuivi] \hfill \\
 Enfin, un dernier cadre vous indique, le cas échéant, vos travaux qui se poursuivent. Ce sont les travaux commencés avant le jour courant et qui arriveront  à échéance un ou plusieurs jours plus tard.



\item[Liens utiles] \hfill \\
 Avant de signer, le message vous indique quelques liens utiles pour suivre le programme.
\end{description}

\subsection{Modifier les préférences concernant le mail quotidien}\hypertarget{modifier-les-prfrences-concernant-le-mail-quotidien}{}\label{modifier-les-prfrences-concernant-le-mail-quotidien}

Vous pouvez modifier le comportement du mail quotidien en modifiant vos \preferences{}. Pour ce faire, rejoignez votre centre de travail et l'onglet \enquote{préférences} puis descendez dans la section \enquote{Notifications}.

Dans cette section \enquote{Notifications}, vous avez deux menus concernant le mail quotidien~{}:

\begin{itemize}
\item si la case \enquote{Récapitulatif journalier} est cochée, quelle que soit la situation de votre programme, quel que soit votre rythme, vous recevez un mail de rapport de cette situation,
\item si la case \enquote{Envoyer le mail quotidien à} est cochée, alors ce mail journalier vous sera envoyé chaque matin, à l'heure que vous choisirez dans le menu.
\end{itemize}
