% <!--
%   CE FICHIER CONTIENT LES DÉFINITIONS GÉNÉRALES DES LIENS
% 
%   DÉFINITION :
% 
%   [un texte identifiant]:  une/path/to/cible  "Le titre optionnel"
% 
%   UTILISATION
% 
%   avec le texte identique :
% 
% un texte identifiant][
% 
%   avec un autre texte :
% 
% autre texte pour le lien][un texte identifiant
% 
%   -->

\chapter{Le programme au quotidien}\hypertarget{programme-quotidien}{}\label{programme-quotidien}

\section{Introduction}\hypertarget{introduction}{}\label{introduction}

La journée d'un\fem{e} auteur\fem{e} qui suit le programme \unan{} commence généralement, après un bon café ou un thé, par la \emph{réception d'un mail} indiquant le travail à accomplir dans  la journée.

C'est ce qui se passe en tout cas si votre rythme est “normal” et que vous avez défini dans vos préférences que vous vouliez recevoir le mail le matin du jour de votre inscription. Dans le cas contraire, vous recevrez le mail très exactement un jour après votre inscription. Si vous vous êtes inscrit\fem{e} à 15 heures, vous recevrez ce message à 15 ou 16 heures chaque jour.

Notez aussi que ce mail peut n'être envoyé qu'aux changements de travaux si vos \hyperlink{preferences-auteur}{préférences} sont réglées dans ce sens. Nous y reviendrons dans la section décrivant votre mail quotidien (\ref{mail-quotidien}).

\section{Prise en compte des travaux}\hypertarget{prise-en-compte-des-travaux}{}\label{prise-en-compte-des-travaux}

Notez un point important~{}: dès qu'un travail est à effectuer, la première chose que vous devez faire est de \textbf{spécifier au programme que vous l'avez vu}. Cela se fait tout simplement en cliquant sur le bouton associé à ce nouveau travail~{}:

\begin{itemize}
\item pour les travaux ou les questionnaires, il s'agit d'un bouton demandant de “démarrer” le travail,
\item pour les pages de cours, il s'agit d'un bouton vous permettant d'indiquer que vous avez “vu” cette page de cours à lire (noter qu'ici le verbe \emph{voir} n'est pas le verbe \emph{lire}~{}; \emph{voir} qu'il n'y a une page de cours à lire ne signifie pas que vous l'avez \emph{lue}).
\end{itemize}
