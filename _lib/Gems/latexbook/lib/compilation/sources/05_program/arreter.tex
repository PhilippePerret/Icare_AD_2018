% <!--
%   CE FICHIER CONTIENT LES DÉFINITIONS GÉNÉRALES DES LIENS
% 
%   DÉFINITION :
% 
%   [un texte identifiant]:  une/path/to/cible  "Le titre optionnel"
% 
%   UTILISATION
% 
%   avec le texte identique :
% 
% un texte identifiant][
% 
%   avec un autre texte :
% 
% autre texte pour le lien][un texte identifiant
% 
%   -->

\section{Mettre le programme en pause}\hypertarget{programme-en-pause}{}\label{programme-en-pause}

Pour le moment, il n'existe aucun moyen de mettre le programme en pause par vous-mêmes, car cela compliquerait considérablement le calcul de votre suivi. En cas d'extrême urgence, vous pouvez néanmoins \mailadmin{contacter l'administration du programme} pour forcer cette opération.

\section{Abandonner le programme}\hypertarget{abandonner-le-programme}{}\label{abandonner-le-programme}

Si pour x ou y raisons vous voulez interrompre définitivement votre programme \unan{}, il vous suffit de rejoindre l'onglet ``État'' sur votre bureau de travail et cliquer sur le bouton ``Abandonner le programme''.

Merci de noter qu'un remboursement ne pourra intervenir que si vous êtes dans \emph{les 20 premiers jours} de ce programme.

